\section{Wirtschaftlichkeit und Investitionsabschätzung}

Ein wesentlicher Bestandteil der Bewertung eines Automatisierungsprojekts ist die wirtschaftliche 
Betrachtung. Neben den technischen Vorteilen müssen insbesondere Investitions- und Betriebskosten gegenübergestellt werden, um den Nutzen der neuen Anlage objektiv beurteilen zu können. 
Im Folgenden werden die Personalkosten des bisherigen Prozesses analysiert, die 
voraussichtlichen Investitionen abgeschätzt und eine grobe Wirtschaftlichkeitsbewertung vorgenommen.

\subsection{Personalkosten: Ist-/Soll-Vergleich}

Der aktuelle Verpackungsprozess wird vollständig manuell durchgeführt und erfordert pro Schicht zwischen drei und vier Mitarbeitende. Dies führt zu hohen Personalkosten und einer 
starken Abhängigkeit vom Arbeitskräfteangebot. Zusätzlich entstehen ergonomische Belastungen, 
die zu Ausfallzeiten und gesundheitlichen Beschwerden führen können.\\

Durch die Einführung einer automatisierten Verpackungsanlage reduziert sich der direkte Personalbedarf 
deutlich. Für den Betrieb der neuen Anlage ist lediglich eine Aufsichtsperson notwendig, 
die den Prozess überwacht und bei Störungen eingreift. 
Dies führt zu einer erheblichen Entlastung der Mitarbeitenden und zu langfristigen Einsparungen 
bei den Betriebskosten.\\

Die folgende Tabelle zeigt den Vergleich des Personalaufwands zwischen der manuellen 
und der automatisierten Variante:

\begin{table}[H]
	\centering
	\begin{tabular}{|p{5cm}|c|c|}
		\hline
		\textbf{Kennzahl} & \textbf{Manuell} & \textbf{Automatisiert} \\ \hline
		Personalbedarf pro Schicht & 3--4 Personen & 1 Person \\ \hline
		Geschätzte jährliche Personalkosten & 100\,\% & ca. 35--40\,\% \\ \hline
		Ergonomische Belastung & Hoch & Sehr gering \\ \hline
		Prozessstabilität & Schwankend & Hoch \\ \hline
	\end{tabular}
	\caption{Personalkostenvergleich zwischen manuellem und automatisiertem Verpackungsprozess}
\end{table}

Der deutlich geringere Personalbedarf führt zu langfristigen Kosteneinsparungen. 
Neben den reinen Lohnkosten entfallen auch Aufwendungen für Schulungen, Einarbeitung, 
ergonomische Arbeitsplatzgestaltung sowie krankheitsbedingte Ausfälle. 
Die Einsparung im Personalbereich stellt somit einen der zentralen wirtschaftlichen Vorteile 
der Automatisierung dar.

\subsection{Grobe Investitionskosten}

Für die Umsetzung der automatisierten Verpackungsanlage fallen einmalige Investitionskosten an, 
die je nach Ausführung und Umfang variieren können. Eine genaue Kostenplanung erfolgt 
im Rahmen eines detaillierten Feinkonzepts. Für das Grobkonzept reichen jedoch 
Schätzwerte, die sich an Erfahrungswerten aus vergleichbaren Projekten orientieren.\\

Die geschätzten Investitionskosten setzen sich aus folgenden Positionen zusammen:

\begin{itemize}
	\item Industrieroboter inkl. Steuerung und Sicherheitsmodulen
	\item Greifertechnik für die Produktaufnahme
	\item Fördertechnik sowie Anpassung der Materialflusskomponenten
	\item Schaltschrank, SPS-Hardware und Netzwerktechnik
	\item Konstruktion, Engineering und Programmierung
	\item Sicherheitszelle bzw. Schutzumhausung
	\item Inbetriebnahme und interne Abnahme
\end{itemize}

Je nach Hersteller und Ausführung ist im Regelfall mit einer Gesamtinvestition 
im unteren bis mittleren sechsstelligen Bereich zu rechnen. 
Ein erheblicher Teil der Investition entfällt dabei auf die sicherheitsrelevanten Komponenten 
wie Lichtgitter, Schutztüren und die mechanische Umhausung der Roboterzelle.

\subsection{Wirtschaftliche Gesamtbewertung und ROI-Betrachtung}

Die wirtschaftliche Bewertung ergibt sich aus dem Verhältnis zwischen Investitionskosten 
und den jährlichen Einsparungen. 
Durch die deutliche Reduzierung des Personalbedarfs und den stabileren Prozess mit geringeren 
Stillstandszeiten können signifikante Kostenvorteile erzielt werden.\newpage

Ein vereinfachter ROI (\textit{Return on Investment}) kann wie folgt dargestellt werden:

\[
ROI = \frac{\text{jährliche Einsparung}}{\text{Investitionskosten}}
\]
\\
Wird beispielsweise von jährlichen Einsparungen von 30--40\,\% der bisherigen Personalkosten 
ausgegangen, amortisiert sich die Anlage typischerweise innerhalb von drei bis fünf Jahren. 
Zusätzlich entstehen qualitative Vorteile wie eine verbesserte Prozessstabilität, 
konstante Verpackungsqualität und eine höhere Anlagenverfügbarkeit, 
die in einer rein finanziellen Betrachtung nicht vollständig abgebildet werden können.\\

Insgesamt zeigt die Wirtschaftlichkeitsbetrachtung, dass die Einführung der automatisierten 
Verpackungsanlage sowohl aus technischer als auch aus finanzieller Sicht sinnvoll ist. 
Die Einsparungen im Betrieb, die steigende Anlagenleistung und die langfristig verbesserte 
Prozessqualität rechtfertigen die Investition deutlich.
