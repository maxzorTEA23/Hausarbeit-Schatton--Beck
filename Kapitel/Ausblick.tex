\section{Ausblick}

Die entwickelte Konzeptlösung bildet eine solide Grundlage für zukünftige Erweiterungen und Optimierungen des Verpackungsprozesses. Durch den modularen Aufbau der Anlage besteht die Möglichkeit, das System bei Bedarf flexibel anzupassen. Dies betrifft sowohl Produktvarianten als auch potenziell steigende Produktionsvolumina.

Die Einbindung in das bestehende MES-System liefert zudem wertvolle Prozessdaten, die künftig zur Optimierung von Abläufen, zur frühen Erkennung von Abweichungen sowie zur Weiterentwicklung der Anlagenperformance genutzt werden können. Dadurch entsteht langfristig ein deutlicher Mehrwert für den gesamten Produktionsbereich.

Darüber hinaus ergeben sich durch die Entlastung der Mitarbeitenden neue Möglichkeiten für deren Qualifizierung und den Einsatz in anspruchsvolleren Tätigkeitsfeldern. Dies trägt nicht nur zur Motivation im Team bei, sondern stärkt auch die zukünftige Kompetenzentwicklung im Unternehmen.

Insgesamt zeigt der Ausblick, dass die Automatisierung nicht nur eine direkte Verbesserung des Verpackungsprozesses darstellt, sondern gleichzeitig den Weg für weitere digitale und technologische Fortschritte ebnet. Das vorgestellte Konzept ist damit ein wichtiger Schritt hin zu einer modernen, leistungsfähigen und langfristig wettbewerbsfähigen Produktionsstruktur.
