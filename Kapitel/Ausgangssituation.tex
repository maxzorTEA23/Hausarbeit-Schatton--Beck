\section{Ausgangssituation}
Derzeit werden die genannten Bremssysteme manuell verpackt. Hierfür steht am Ende der Produktionslinie ein separater Bereich zur Verfügung. In diesem sind pro Schicht etwa drei bis vier Mitarbeitende eingeplant, die die fertigen Produkte per Hand verpacken.\\
Dieser manuelle Prozess ist mit einem hohen Zeitaufwand verbunden. Dadurch sinkt die Zykluszeit des Gesamtprozesses, und es kann keine durchgehend konstante Verpackungsqualität gewährleistet werden.\\
Während der Automatisierungsgrad der übrigen Produktionslinie sehr hoch ist, reduziert der manuelle Verpackungsprozess diesen Wert deutlich. Dies wirkt sich spürbar auf die Auslastung und die Quantität der Linie aus.\\
Darüber hinaus entstehen für das Unternehmen hohe Personalkosten, die unter anderem folgende Punkte umfassen:
\begin{itemize}
	\item Gehaltskosten inklusive Bonuszahlungen
	\item Sozialabgaben und Zusatzleistungen
	\item Schulungs- und Einarbeitungskosten
\end{itemize}
Auch für die Mitarbeitenden ist der Verpackungsprozess mit einer hohen körperlichen und psychischen Belastung verbunden. Die Tätigkeit ist eintönig und bietet kaum Möglichkeiten zur Weiterentwicklung oder Qualifizierung.\\
Das größte Defizit liegt jedoch in der Entstehung eines Engpasses im Produktionsfluss, wodurch die Gesamtleistung der Linie beeinträchtigt wird. Infolgedessen kann die Wettbewerbsfähigkeit des Unternehmens langfristig gefährdet sein.
