\section{Fazit}

Die vorliegende Arbeit zeigt, dass die Einführung einer automatisierten Verpackungsanlage einen deutlichen Mehrwert für den bestehenden Produktionsprozess darstellt. Durch den Einsatz eines Industrieroboters, einer angepassten Fördertechnik sowie der zentralen Beckhoff-SPS entsteht ein stabiler und effizienter Prozess, der sich nahtlos in die bestehende Produktionslinie integrieren lässt. 

Die Analyse der Ausgangssituation verdeutlicht, dass insbesondere die ergonomische Belastung, die hohen Personalkosten sowie die schwankende Verpackungsqualität zentrale Herausforderungen des bisherigen manuellen Prozesses darstellen. Diese Defizite können durch die geplante Automatisierung effektiv reduziert werden.

Ein wesentlicher Vorteil ergibt sich durch die MES-Anbindung, die eine durchgängige Dokumentation, Prozessüberwachung und Auswertung von Produktionskennzahlen ermöglicht. Damit wird ein wichtiger Schritt in Richtung einer digitalisierten und datenbasierten Produktionsumgebung umgesetzt. 

Auch aus wirtschaftlicher Sicht zeigt sich, dass die Automatisierung langfristige Einsparungen ermöglicht. Trotz der erforderlichen Investitionen führt die Reduzierung des Personalbedarfs sowie die höhere Prozessstabilität zu einer verbesserten Wirtschaftlichkeit. Insgesamt bestätigt das entwickelte Konzept, dass die Umsetzung technisch sinnvoll, wirtschaftlich tragfähig und prozessseitig klar vorteilhaft ist.
