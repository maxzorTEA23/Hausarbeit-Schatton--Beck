\section{Zielsetzung und Nutzen(Benefits)}
Ziel des Grobkonzepts ist es eine vollständige automatisierte Anlage zu entwickeln. Diese soll die, aktuell noch durchgeführten manuellen Tätigkeiten zur Verpackung der Bremssysteme, vollständig als eigenständig integriertes System ablösen.\\
Durch die Implementierung einer solch automatisierten Anlage soll der Gesamtprozess effizienter gestaltet werden. Ein weiteres Ziel ist die Stabilisierung einer gleichbleibend hohen Qualität. Somit werden Engpässe im Produktionsfluss vermieden und die Anlagenverfügbarkeit steigt.\\\\
Ein zentrales Ziel besteht darin die Zykluszeit zu verringern und damit den Durchsatz\footnote{text} zu erhöhen. Somit kann die Gewinnmaximierung garantiert werden. Zudem soll das System eine reproduzierbare Verpackungsqualität gewährleisten und somit Fehlereinflüsse, welche durch die Bearbeitung der Menschen entstehen, vermeiden.\\\\
Neben den technischen Vorteilen gibt es auch wirtschaftliche und ergonomische Aspekte, welche den Stakeholder bei der Entscheidungsfindung helfen sollen. Der Personalbedarf kann drastisch gesenkt werden. Somit ist eine langfristige Reduzierung von Kosten gewährleistet. Gleichzeit werden die Mitarbeiter entlastet. Ein Einsatz der Werker in einem höherwertigen Tätigkeitsfeld ist dadurch möglich, das führt langfristig zu einer höheren Motivation und Qualifizierung. 