\section{Innovationsgrad}

Der Innovationsgrad des vorgestellten Konzepts ergibt sich sowohl aus technologischen als auch aus organisatorischen Fortschritten. Obwohl einzelne eingesetzte Technologien – wie Industrieroboter, Fördertechnik oder SPS-Steuerungen – bereits etabliert sind, entsteht der innovative Charakter insbesondere durch deren spezifische Kombination, die Integration in die bestehende Produktionsumgebung sowie den erzielten Mehrwert für den gesamten Prozessfluss.

\subsection{Technologischer Innovationsgrad}

Ein zentraler innovativer Aspekt ist die vollständige Automatisierung eines bislang rein manuell durchgeführten Verpackungsprozesses\cite{ABB_Applications}. Während robotergestützte Montageprozesse in der Produktion weit verbreitet sind, stellt die automatische Verpackung komplexer Bremssysteme eine besondere Herausforderung dar.

Die Kombination aus:
\begin{itemize}
	\item \textbf{ABB-Industrieroboter} mit anwendungsspezifischer Greifertechnik,
	\item \textbf{Beckhoff-SPS} als zentrale Steuerungseinheit,
	\item \textbf{modularer Fördertechnik} und Werkstückhandhabung,
	\item und \textbf{standardisierten Schnittstellen} (EtherCAT, Profinet, OPC~UA)
\end{itemize}

führt zu einer modernen, flexiblen und zukunftssicheren Lösung. Damit entsteht ein deutlicher Innovationssprung im Vergleich zur aktuellen Situation, in der alle Handlings- und Verpackungsprozesse manuell durchgeführt werden.

\subsection{Digitalisierung und Datenintegration}

Die Anbindung der neuen Verpackungsanlage an das bestehende Manufacturing Execution System (MES) erhöht den Innovationsgrad zusätzlich. Durch die Bereitstellung von Echtzeitdaten wie Stückzahlen, Anlagenzuständen, Fehlermeldungen oder Prozessparametern entsteht ein digital transparenter Verpackungsprozess.

\subsubsection*{OEE-Kennwert (Overall Equipment Effectiveness)}

Im Rahmen der MES-Anbindung spielt der OEE-Kennwert (Overall Equipment Effectiveness) eine zentrale Rolle.\newpage
Er dient dazu, die Gesamteffizienz einer Anlage zu bewerten und setzt sich aus drei Hauptfaktoren zusammen:

\begin{itemize}
	\item \textbf{Verfügbarkeit} – tatsächliche Laufzeit der Anlage im Verhältnis zur geplanten Produktionszeit,
	\item \textbf{Leistung} – Abgleich zwischen realer und theoretisch möglicher Geschwindigkeit,
	\item \textbf{Qualität} – Anteil der fehlerfreien Produkte an der Gesamtmenge.
\end{itemize}

Die Berechnung erfolgt nach der Formel:

\[
OEE = \text{Leistung} \cdot \text{Qualität} \cdot \text{Verfügbarkeit}
\]

Durch die automatisierte Datenerfassung mittels OPC~UA können erstmals alle relevanten Prozesskennzahlen zuverlässig und kontinuierlich an das MES gemeldet werden\cite{OPCFoundation}. Dies war im bisherigen manuellen Prozess nicht möglich. Dadurch wird eine objektive Bewertung des Anlagenzustands und der Prozessperformance ermöglicht – ein wesentlicher Schritt in Richtung Industrie~4.0.

\subsection{Organisatorische und wirtschaftliche Innovation}

Neben den technischen Neuerungen bietet das Konzept auch organisatorische Vorteile. Die Automatisierung entlastet Mitarbeitende von monotonen und ergonomisch belastenden Tätigkeiten und ermöglicht eine qualifikationsgerechtere Aufgabenverteilung. Dies steigert sowohl Motivation als auch langfristige Zufriedenheit.

Durch:
\begin{itemize}
	\item stabilisierte \textbf{Zykluszeiten},
	\item reproduzierbare \textbf{Verpackungsqualität},
	\item reduzierten \textbf{Personalaufwand} 
\end{itemize}

ergibt sich eine deutliche wirtschaftliche Verbesserung. Die systematische Datentransparenz ermöglicht darüber hinaus eine fortlaufende Optimierung der gesamten Produktionskette.

\subsection{Gesamtbewertung}

Insgesamt weist das vorgestellte Konzept einen mittleren bis hohen Innovationsgrad auf. Zwar kommen etablierte Technologien zum Einsatz, jedoch in einer neuartigen Konfiguration und tiefen Integration in das bestehende Produktionssystem. Die digitale Vernetzung über OPC~UA und die umfassende MES-Anbindung erhöhen den Fortschritt zusätzlich.

Damit entsteht ein skalierbares, zukunftsorientiertes und prozessoptimierendes System, das die Wettbewerbsfähigkeit des Unternehmens nachhaltig stärkt.
