\section{Risikoanalyse und Stakeholderbewertung}

Die erfolgreiche Umsetzung der automatisierten Verpackungsanlage erfordert eine 
Betrachtung möglicher technischer und organisatorischer Risiken. 
Durch eine frühzeitige Identifikation dieser Risiken können geeignete Maßnahmen 
entwickelt werden, um den Projektverlauf stabil und planbar zu gestalten. 
Darüber hinaus ist es notwendig, die beteiligten Stakeholder zu betrachten, 
da diese maßgeblichen Einfluss auf Entscheidungsprozesse, Projektfortschritt 
und spätere Nutzung des Systems haben.

\subsection{Technische Risiken}

Im Rahmen der Automatisierung ergeben sich verschiedene technische Risiken, 
die während Planung, Aufbau und Betrieb berücksichtigt werden müssen:

\begin{itemize}
	\item \textbf{Greifer- und Handhabungsprobleme:} 
	Geometrische Abweichungen oder variierende Bauteiltoleranzen können zu Fehlgriffen 
	oder Fehllagen des Produkts führen.
	
	\item \textbf{Sensorische Fehlmessungen:} 
	Verschmutzungen, Fremdlicht oder Verschleiß können die Genauigkeit von 
	Lichtschranken oder Positionssensoren beeinflussen.
	
	\item \textbf{Ausfall von Aktoren oder Fördertechnik:} 
	Mechanischer Verschleiß, blockierte Transportbänder oder Ausfälle der 
	Antriebstechnik können den Prozessfluss stören.
	
	\item \textbf{Kommunikationsfehler:} 
	Unterbrechungen oder Fehltelegramme im Profinet- bzw. EtherCAT-Netzwerk 
	können den Prozessablauf beeinträchtigen.
	
	\item \textbf{Sicherheitsrisiken:} 
	Fehlerhafte Signale an sicherheitsrelevante Komponenten (z. B. Schutztüren, Lichtgitter) 
	können ungeplante Stillstände auslösen.
\end{itemize}

\subsection{Organisatorische Risiken}

Neben technischen Risiken spielen organisatorische Aspekte eine wichtige Rolle:

\begin{itemize}
	\item \textbf{Schulungsbedarf:} 
	Bediener und Instandhalter müssen im Umgang mit Robotertechnik und 
	Programmierschnittstellen geschult werden.
	
	\item \textbf{Akzeptanz der Mitarbeitenden:}  
	Veränderungsprozesse können anfänglich zu Unsicherheit oder Ablehnung führen, 
	insbesondere wenn Arbeitsaufgaben entfallen oder sich verändern.
	
	\item \textbf{Projektabhängigkeiten:} 
	Die Umsetzung hängt von internen Abteilungen wie Instandhaltung, 
	Industrial Engineering oder IT ab, wodurch Verzögerungen entstehen können.
	
	\item \textbf{Integration in den Schichtbetrieb:}  
	Neue Abläufe müssen an bestehende Materialströme und Schichtmodelle angepasst werden.
\end{itemize}

\subsection{Stakeholderanalyse}

Für die erfolgreiche Einführung der automatisierten Verpackungsanlage ist es 
essenziell, die beteiligten Interessengruppen und deren Anforderungen zu berücksichtigen.\\
Für die Bewertung des Projekterfolgs ist es entscheidend, die unterschiedlichen 
Blickwinkel der beteiligten Interessengruppen einzubeziehen. Jede Gruppe bringt 
eigene Anforderungen, Erwartungen und Prioritäten mit, die sich unmittelbar auf 
Planung, Umsetzung und späteren Betrieb der Anlage auswirken. Besonders bei 
Automatisierungsprojekten, die sowohl technische als auch organisatorische 
Veränderungen mit sich bringen, ist eine strukturierte Betrachtung der Stakeholder 
von zentraler Bedeutung. Nur wenn die jeweiligen Bedürfnisse frühzeitig erkannt 
und berücksichtigt werden, kann ein reibungsloser Projektverlauf sichergestellt 
werden. Dies betrifft sowohl die technische Umsetzung als auch Aspekte wie 
Akzeptanz, Schulung und langfristige Integration in bestehende Prozesse.\newpage
Die folgende Tabelle gibt einen Überblick über die wichtigsten Stakeholder:

\begin{table}[H]
	\centering
	\begin{tabular}{|p{4cm}|p{9cm}|}
		\hline
		\textbf{Stakeholder} & \textbf{Interessen / Aufgaben} \\ \hline
		Geschäftsführung & Wirtschaftlichkeit, Investitionsentscheidung, Effizienzsteigerung \\ \hline
		Produktion & Stabile Abläufe, einfache Bedienung, höhere Durchsatzleistung \\ \hline
		Instandhaltung & Wartungszugänglichkeit, Diagnosemöglichkeiten, robuste Technik \\ \hline
		Qualitätsmanagement & Reproduzierbare Verpackungsqualität, klare Prozessdokumentation \\ \hline
		Industrial Engineering & Planung, Simulation und Optimierung des Anlagenkonzepts \\ \hline
		IT / MES-Abteilung & Sichere Datenübertragung, Kompatibilität zur MES-Architektur \\ \hline
		Mitarbeitende am Arbeitsplatz & Ergonomische Entlastung, klare Bedienkonzepte, Schulung \\ \hline
	\end{tabular}
	\caption{Stakeholderübersicht}
\end{table}

Die Zusammenführung der Stakeholderinteressen bildet eine wichtige Grundlage 
für die erfolgreiche Umsetzung des Projekts. Durch eine transparente Kommunikation 
und die frühzeitige Einbindung aller beteiligten Bereiche können Risiken minimiert 
und Akzeptanz geschaffen werden.
