\section{Einleitung}
Im Rahmen der kontinuierlichen Prozessoptimierung in der Produktion, beim Unternehmen \textbf{Aumovio SE}, wurde ein erhöhtes Automatisierungspotenzial bei der Verpackung von Bremssystemen festgestellt.
Derzeit erfolgt dieser Prozess noch manuell durch Mitarbeitende der Fertigung. Dies führt unter anderem zu wirtschaftlichen Nachteilen bzw. zusätzlichen Kosten sowie zu potenziellen gesundheitlichen Belastungen für die Mitarbeitenden.\\
Zudem gestaltet es sich zunehmend schwierig, die Verpackungsqualität an den steigenden Produktionsdurchsatz anzupassen.\\\\
Aufgrund der genannten Punkte wird im Folgenden ein Grobkonzept entwickelt, das der Weiterentwicklung der Automatisierung des beschriebenen Prozesses dient.\\\\
Ziel des Konzepts ist es, die bestehenden Defizite zu beseitigen und eine Steigerung der Qualität und Effizienz der Verpackungsstation zu ermöglichen. Es soll ein automatisiertes System entstehen, das den manuellen Prozess ablöst und sich nahtlos in die bestehende Produktionslinie integrieren lässt. Durch den Einsatz moderner Robotertechnik und automatisierter Fördertechnik können eine gleichbleibende Qualität, eine Reduzierung der Prozesszeiten sowie eine Entlastung des Personals erreicht werden.\\\\
Das Konzept dient der Geschäftsführung als Entscheidungsgrundlage, um den Nutzen und die Wirtschaftlichkeit einer automatisierten Verpackungsanlage zu bewerten. Dabei werden neben den technischen Vorteilen auch wirtschaftliche Aspekte, der Innovationsgrad sowie die erforderlichen Investitionen betrachtet. Langfristig soll die Anlage dazu beitragen, die Produktivität zu steigern und die Wettbewerbsfähigkeit durch eine kontinuierlich hohe Kundenzufriedenheit zu erhöhen.
