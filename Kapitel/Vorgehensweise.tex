\section{Vorgehensweise des Projekts}

Für die erfolgreiche Umsetzung der automatisierten Verpackungsanlage ist ein strukturiertes
Vorgehen notwendig. Das Projekt wird in mehrere aufeinander aufbauende Phasen unterteilt,
um technische, organisatorische und wirtschaftliche Aspekte systematisch zu berücksichtigen.
Durch eine klare Aufteilung der Arbeitsschritte können Risiken frühzeitig erkannt und
Entscheidungen fundiert getroffen werden. Die nachfolgenden Abschnitte beschreiben die
empfohlene Vorgehensweise.

\subsection{Analysephase}

Zu Beginn des Projekts steht die detaillierte Analyse des bestehenden Verpackungsprozesses.
Hierbei werden alle relevanten Arbeitsschritte, Taktzeiten, ergonomische Belastungen sowie
Störquellen dokumentiert. Parallel werden Anforderungen aus Produktion, Instandhaltung und
Qualitätsmanagement aufgenommen. Diese Phase bildet die Grundlage für die spätere technische
Auslegung der Anlage. Außerdem werden Kommunikationsschnittstellen, Sicherheitsanforderungen
und Messwerte identifiziert, die für die spätere Automatisierung notwendig sind.

\subsection{Konzeption und Schnittstellendefinition}

Auf Basis der Analyse erfolgt die Erstellung des Grob- und Feinkonzepts. Dazu gehören die
Festlegung der Bewegungsabläufe des Roboters, die Auswahl geeigneter Greifertechnik sowie die
Definition aller Ein- und Ausgangssignale. In dieser Phase wird ebenfalls die Architektur der
Steuerung festgelegt, insbesondere die Kommunikation über EtherCAT, Profinet und die
Anbindung an das MES-System. Zusätzlich werden Layouts der Roboterzelle, Sicherheitsbereiche
und notwendige Komponenten der Fördertechnik festgelegt. Diese Phase dient als Grundlage für
die Freigabe durch die Geschäftsführung.

\subsection{Umsetzung und Realisierung}

Nach der Konzeptfreigabe beginnt die technische Umsetzung. Dazu gehören die mechanische
Konstruktion der Aufnahmen, der Aufbau der Sicherheitsumhausung sowie die Installation der
Felderfassung und Antriebstechnik. Parallel entwickeln SPS- und Roboterprogrammierung die
Steuerlogik, Bewegungsabläufe und Sicherheitsfunktionen. Für die SPS wird die
Signalverarbeitung implementiert, während der Roboter mit den notwendigen Pick-and-Place
Bewegungen parametriert wird. Die Kommunikation aller Systeme wird in dieser Phase getestet
und schrittweise in Betrieb genommen.

\subsection{Testphase, Optimierung und Übergabe}

Nach Abschluss der Umsetzung folgt eine Test- und Optimierungsphase. In dieser Phase wird der
vollständige Verpackungsprozess unter realen Bedingungen geprüft. Zykluszeiten, Greifqualität
und Prozessstabilität werden analysiert und angepasst. Ebenso wird die Kommunikation mit dem
MES-System sowie die Reaktion auf Störfälle getestet. Abschließend erfolgt die Übergabe der
Anlage an die Produktion, einschließlich Schulung des Bedienpersonals, Übergabe der
Dokumentation und Freigabe durch das Qualitätsmanagement.

Durch diese strukturierte Vorgehensweise kann sichergestellt werden, dass die Anlage technisch
zuverlässig aufgebaut wird, wirtschaftliche Ziele erreicht werden und der Produktionsprozess
langfristig stabil betrieben werden kann.
